\documentclass{article}

\usepackage[TU]{fontenc}
\usepackage[english, russian]{babel}
\usepackage{fevcard}
\usepackage{xifthen}
\usepackage{datatool}
\usepackage[paperheight=3.5in,paperwidth=2.5in,margin=0pt]{geometry}
\setlength\parindent{0pt}
\usepackage{fontspec}
\renewcommand{\familydefault}{\sfdefault}
\usepackage{libertine}

\begin{document}
\exhyphenpenalty 10000
\hyphenpenalty 10000
\DTLsetseparator{,}
\DTLloaddb{cards}{../cards.csv}

\newcounter{countnum}
\newcounter{countdir}
\setcounter{countnum}{0}
\setcounter{countdir}{0}

\DTLforeach*
{cards}% database label
{\csvtype=type,\csvcardtext=cardtext,\csvspec=spec}
{
 \ifthenelse{\not\equal{\csvcardtext}{SKIP}}{
  \ifthenelse{\equal{\csvspec}{EVENT}}
  {\def\csvdirection{EVENT}}%это событие
  {\def\csvdirection{ARROW}}
  \begin{tcard}{fev}%
    \cardtext{\csvcardtext}%
	\type{\csvtype}%
    \randomdir{\csvdirection}%
\end{tcard}
}
{}
}\end{document} %не добавляй пробелов внутри цикла и после него