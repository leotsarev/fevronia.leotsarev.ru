\documentclass{article}

\usepackage[TU]{fontenc}
\usepackage[english, russian]{babel}
\usepackage{fevcard}
\usepackage{xifthen}
\usepackage{datatool}
\usepackage[a4paper,margin=0.4in]{geometry}
\setlength\parindent{0pt}
\usepackage{fontspec}
\renewcommand{\familydefault}{\sfdefault}
\usepackage{libertine}

\begin{document}
\exhyphenpenalty 10000
\hyphenpenalty 10000
\DTLsetseparator{,}
\DTLloaddb{cards}{../cards.csv}

\newcounter{countnum}
\setcounter{countnum}{0}

\DTLforeach*
{cards}% database label
{\csvtype=type,\csvcardtext=cardtext,\csvspec=spec}
{
  \ifthenelse{\equal{\csvspec}{EVENT}}
  %это событие
  {\def\csvdirection{EVENT}}
  {% Определяем направление (L/R) по очереди
    \DTLiffirstrow{\def\csvdirection{R}}{% 
    \DTLifoddrow{\def\csvdirection{R}}{\def\csvdirection{L}}%
  }
  }
  % Определяем число (1/2/3) по очереди
  \stepcounter{countnum}%
  \ifnum\value{countnum}=1
    \def\csvcount{1}%
  \else\ifnum\value{countnum}=2
    \def\csvcount{2}%
  \else
    \def\csvcount{3}%
    \setcounter{countnum}{0} % Сбрасываем счетчик после 3
  \fi\fi
  \begin
  {tcard}{fev}%
    \cardtext{\csvcardtext}%
	\type{\csvtype}%
    \randomdir{\csvdirection}
    \randomcount{\csvcount}
\end{tcard}
}

\end{document}